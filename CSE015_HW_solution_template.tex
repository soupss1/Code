\documentclass[11pt]{article}

% To produce a letter size output. Otherwise will be A4 size.
\usepackage[letterpaper]{geometry}

% For enumerated lists using letters: a. b. etc.
\usepackage{enumitem}

\usepackage{amsmath}



\topmargin -.5in
\textheight 9in
\oddsidemargin -.25in
\evensidemargin -.25in
\textwidth 7in

\begin{document}

% Edit the following putting your first and last names and your lab section.
\author{Angel Guzman\\
Lab CSE-015-03L T 7:30-10:20pm}

% Edit the following replacing X with the HW number.
\title{CSE 015: Discrete Mathematics\\
Fall 2019\\
Homework \#2\\
Solution}

% Put today's date in the following.
\date{September 30, 2024
}
\maketitle

% ========== Begin questions here
\begin{enumerate}

\section{\textbf{Tautologies, Contingencies, and Contradictions}}
\item 

\textbf{Question 1: (($p \implies q$)$\land p$) $\implies q$}

\begin{enumerate}[label=(\alph*)]
\item
\begin{center}
\begin{tabular}{ |c|c|c|c|c| }
 \hline
 p & q & $p \implies q$ & ($p \implies q$)$\land p$) & ($p \implies q$)$\land p$) $\implies q$ \\
 \hline
 F & F & T & F & T \\
 F & T & T & F & T \\
 T & F & F & F & T \\
 T & T & T & T & T \\
 \hline

\end{tabular}
\end{center}

\item 
This is a tautology as the truth table shows that no matter the truth values of the propositional variables the compound proposition will always be true.

\end{enumerate}

\textbf{Question 2: ($\neg r$) $\land$ ($q \implies r$) $\land q$}

\begin{enumerate}[label=(\alph*)]
\item
\begin{center}
\begin{tabular}{ |c|c|c|c|c|c| }
 \hline
 r & q & $\neg r$ & $q \implies r$ & ($\neg r$) $\land$ ($q \implies r$) & (($\neg r$) $\land$ ($q \implies r$)) $\land q$ \\
 \hline
 F & F & T & T & T & F \\
 F & T & T & F & F & F \\
 T & F & F & T & F & F \\
 T & T & F & T & F & F \\
 \hline

\end{tabular}
\end{center}
\item 
This is a contradiction as no matter the truth values of the propositional variables the compound proposition will always be false

\end{enumerate}

\textbf{Question 3: ($ p \land q$) $\land $ ($\neg r \lor \neg q \lor \neg p$)}

\begin{enumerate}[label=(\alph*)]
\item
\begin{center}
\begin{tabular}{ |c|c|c|c|c|c|c|c| }
 \hline
 p & q & r & $\neg p$ & $\neg q$ & $\neg r$ & $ p \land q$ & $\neg r \lor \neg q \lor \neg p$ \\
 \hline
 T & T & T & F & F & F & T & F \\
 F & T & T & T & F & F & F & T \\
 T & F & T & F & T & F & F & T \\
 F & F & T & T & T & F & F & T \\
 T & T & F & F & F & T & T & T \\
 F & T & F & T & F & T & F & T \\
 T & F & F & F & T & T & F & T \\
 F & F & F & T & T & T & F & T \\
 \hline

\end{tabular}
\end{center}
\item 
This compound proposition is a contingency as it is neither a tautology nor a contradiction.

\end{enumerate}

\section{\textbf{Inferences and Tautologies}}

\textbf{Question 1: $(( p \lor q) \land (\neg p \lor r)) \implies (q \lor r)$}

\begin{enumerate}[label=(\alph*)]
\item
\begin{center}
\begin{tabular}{ |c|c|c|c|c|c|c|c|c| }
 \hline
 p & q & r & $\neg p$ & $p \lor q$ & $\neg p \lor r$ & $(( p \lor q) \land (\neg p \lor r))$ & $q \lor r$ & $(( p \lor q) \land (\neg p \lor r)) \implies (q \lor r)$  \\
 \hline
 T & T & T & F & T & T & T & T & T \\
 F & T & T & T & T & T & T & T & T \\
 T & F & T & F & T & T & T & T & T \\
 F & F & T & T & F & T & F & T & T \\
 T & T & F & F & T & F & F & T & T \\
 F & T & F & T & T & T & T & T & T \\
 T & F & F & F & T & F & F & F & T \\
 F & F & F & T & F & T & F & F & T \\
 \hline

\end{tabular}
\end{center}
\item 
This is a tautology as the truth table shows that no matter the truth values of the propositional variables the compound proposition will always be true.

\end{enumerate}

\textbf{Question 2: $(\neg q \land (p \implies q)) \implies \neg p$}

\begin{enumerate}[label=(\alph*)]
\item
\begin{center}
\begin{tabular}{ |c|c|c|c|c|c|c|c|c| }
 \hline
 p & q & $\neg p$ & $\neg q$ & $p \implies q$ & $\neg q \land (p \implies q)$ & $(\neg q \land (p \implies q)) \implies \neg p$ \\
 \hline
 T & T & F & F & T & F & T \\
 F & T & T & F & T & F & T \\
 T & F & F & T & F & F & T \\
 F & F & T & T & T & T & T \\

 \hline

\end{tabular}
\end{center}
\item 
This is a tautology as the truth table shows that no matter the truth values of the propositional variables the compound proposition will always be true.

\end{enumerate}

\section{\textbf{Finite Domains}}
\textbf{Question 1: $\forall x(P(x) \lor Q(x)) $;}
\begin{enumerate}[label=(\alph*)]
\item
$(P(a) \lor Q(a)) \land (P(b) \lor Q(b)) \land (P(c) \lor Q(c)) \land (P(d) \lor Q(d))$

\end{enumerate}

\textbf{Question 2: $\neg \exists \neg P(x)$;}
\begin{enumerate}[label=(\alph*)]
\item
$\neg P(a) \land \neg P(b) \land \neg P(c) \land \neg P(d)$

\end{enumerate}

\textbf{Question 3: $\forall x(P(x) \land \neg Q(x)) $;}
\begin{enumerate}[label=(\alph*)]
\item
$(P(a) \land \neg Q(a)) \land (P(b) \land \neg Q(b)) \land (P(c) \land \neg Q(c)) \land (P(d) \land \neg Q(d)) \land$

\end{enumerate}

\section{\textbf{Quantifiers and Nested Quantifiers}}
\textbf{Question 1: Every number is the opposite of another number}
\begin{enumerate}[label=(\alph*)]
\item
$\forall x(\exists$(y = -x))

\end{enumerate}

\textbf{Question 2: Every positive number is the square of another number}
\begin{enumerate}[label=(\alph*)]
\item
$\forall x(\exists (x^2 = |y|))$

\end{enumerate}

\textbf{Question 3: There is no number whose square is negative}
\begin{enumerate}[label=(\alph*)]
\item
$\forall x(\neg \exists (x^2 < 0 ))$

\end{enumerate}

\textbf{Question 4: The product of any two negative numbers is positive}
\begin{enumerate}[label=(\alph*)]
\item
$\forall x \forall y(-x * -y \implies |z|)$

\end{enumerate}

\section{\textbf{Negating Formulas with Nested Quantifiers}}
\textbf{Question 1: $\exists x \exists y (P(x) \implies Q(y))$}
\begin{enumerate}[label=(\alph*)]
\item
$\forall x \forall y(-P(x) \implies \neg Q(y))$

\end{enumerate}

\textbf{Question 2: $\exists y (\exists x A(x, y) \lor \forall x B(x, y))$}
\begin{enumerate}[label=(\alph*)]
\item
$\forall y (\forall x \neg A(x, y) \lor \exists x \neg B(x, y))$

\end{enumerate}

\end{enumerate}

\end{document}
